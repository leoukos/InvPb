\documentclass[a4paper,10pt]{article}
\usepackage[utf8]{inputenc}
\usepackage[frenchb]{babel}
\usepackage[T1]{fontenc}

\usepackage{authblk}
\usepackage{frbib}
\usepackage{amsfonts}
\usepackage{amsmath}
\usepackage{amssymb}
\usepackage{array}
\usepackage{url}
\usepackage{hyperref}
\usepackage{textcomp}
\usepackage[all]{hypcap}
\usepackage[labelseparator=endash]{caption}
\usepackage[listofformat=parens]{subfig}
\usepackage{graphicx, color}
\usepackage{listings}
\usepackage{float}
\bibliographystyle{frplain}
\hypersetup{colorlinks,%
            citecolor=black,%
            filecolor=black,%
            linkcolor=black,%
            urlcolor=blue}

\newcommand{\LSTRESET}{
  \lstset{
    language=,
    basicstyle=\color{black},
    identifierstyle=\color{black}
}
}

\setlength{\affilsep}{0.2cm}
\author{François Deslandes}
\affil{Génie Mathématique 5ème année}
\affil{A l'attention de M. Caputo}
%\date{4 Décembre 2013\\}
\title{\Huge{Analyse non linéaire et problèmes inverses}\\
\LARGE{Problèmes inverses liés à la cristallisation de polymères.}\\
\vspace{10mm}
}

\DeclareMathOperator{\Rec}{Rec}
\DeclareMathOperator{\card}{card}

\begin{document}
\maketitle\thispagestyle{empty}
 
\newpage\null\thispagestyle{empty}\setcounter{page}{0}

\newpage
\tableofcontents
\newpage

\section{Modèle de base et problème d'identification}

Le problème modélisant la cristallisation de polymères est le suivant :
\[
\left\{
\begin{array}{r c l}
\frac{\partial T}{\partial t} &=& 
			D \frac{\partial ^2T}{dx ^2} 
			+ L \frac{\partial \xi}{\partial t} \\
					
\frac{\partial }{\partial t} (\frac{1}{\widetilde{G}(T) (1-\xi)}) &=& 
			\frac{\partial}{dx} (\frac{1}{\widetilde{G}(T) (1-\xi)})
			+ 2 \frac{\partial }{\partial t}(\widetilde{N}(T))
\end{array}
\right.
\]

Avec les conditions initiales suivantes :
\[
\left\{
\begin{array}{r c l}
T(x,0)		&=& 		T^ 0(x)		\\
\xi(x,0)		&=&		0			\\
\frac{\partial\xi}{\partial t}(x,0) 		&=&		0		\\
\end{array}
\right.
\]

et les conditions aux bords :
\[
\left\{
\begin{array}{r c l r}
\frac{\partial T}{\partial n}(x, t)		&=&		\alpha (T(x,t) -T^1(x,t)) & x \in \partial \Omega\\

\frac{\partial T}{\partial t}(x, t)	+ \widetilde{G}(T) \frac{\partial T}{\partial n}(x, t)	&=&		0	 & x \in \partial \Omega\\


\end{array}
\right.
\]

On cherche à identifier $\widetilde{N}(T)$,le taux de nucléation, connaissant $\widetilde{G}(T)$, le taux de croissance radial, $\xi(x, t_*)$, le degré de cristallisation, et $T$ la température sur les bords du domaine.

\section{Modèle simplifié  et problème d'identification}

Par symétrie et avec des changements de variables, le problème devient :
\[
\left\{
\begin{array}{r c l}
u_t		&=&		D u_{xx} + L e^{-v} v_t		\\
v_t		&=&		a(u)w						\\
w_t		&=&		(a(u)v_x)_x + b(u)_t			\\
\end{array}
\right.
\]

Avec les conditions initiales suivantes :
\[
\left\{
\begin{array}{r c l}
u|_{t=0}		&=&		u^0		\\
v|_{t=0}		&=&		0		\\
w|_{t=0}		&=&		0		\\
\end{array}
\right.
\]

et les conditions aux bords :
\[
\left\{
\begin{array}{r c l c}
u_x(0, t)	&=&		-\alpha (u(0,t) - u^1(0, t)) 	\\
w(0, t)		- v_x(0, t)	&=&		0				 	\\
u_x(1,t)		&=&		0								\\
v_x(1,t)		&=&		0								\\
\end{array}
\right.
\]


Dans ce problème, on cherche $b(u)$, l'équivalent de $\widetilde{N}(T)$. Il peut \^etre exprimé, avec l'opérateur non linéaire $F$, sous la forme suivante :
\[
\begin{array}{r l c}
F(b)		&=&		(u^{\delta}_{B}, v^{\delta}_{*}) \\
\end{array}
\]

Où $u^{\delta}_{B}$ et $v^{\delta}_{*}$ sont des mesures bruitées.


\section{Méthode de résolution}

Pour trouver $b(u)$, on utilise l'itération de Landweber.
\[
\begin{array}{r l c}
b^{k+1} 	&=&	b^{k} +  \omega F'(b^k)*(F(b^k) - (u^{\delta}_{B}, v^{\delta}_{*}) \\
\end{array}
\]

Où $F'$ représente la dérivée de l'opérateur $F$ et $F*$ son adjoint.\\

Par la suite, un algorithme itératif basé en partie sur le calcul de l'adjoint de $F$ sera donnée pour trouver $b$ à partir d'un choix arbiraire de $b^0$.

\subsection{Linéarisé}

Dans un premier temps, l'opérateur $F$ est séparé en : $F=\Psi \circ  \Phi$. L'opérateur $\Psi$ se charge de conditions limites et est linéaire, ce qui permet de dire que : $F'=\Psi \circ  \Phi '$. On suppose que $\Phi '$ existe.
On va obtenir $F'(b)h=\Psi \circ  \Phi'(b) h$ en linéarisant le système. Pour cela, on suppose que $(u,v,w)$ est solution. On suppose ensuite que $(u+U, v+V, w+W)$ est proche de la solution. Avec $a(u+U)=a(u)+a'(u)U$ et $b(u+U)=b(u)+b'(u)U+h(u)$, $(U,V,W)=\Phi'(b) h$ satisfait le système suivant :

\[
\left\{
\begin{array}{r c l}
U_t		&=&		D U_{xx} + L e^{-v} (v_t V +V_t)		\\
V_t		&=&		a(u)W + a'(u)Uw						\\
W_t		&=&		(a(u)V_x)_x + (a'(u)Uv_x)_x + (b'(u)U)_t + h(u)_t v\\
\end{array}
\right.
\]

Avec les conditions initiales suivantes :
\[
\left\{
\begin{array}{r c l}
u|_{t=0}		&=&		0		\\
v|_{t=0}		&=&		0		\\
w|_{t=0}		&=&		0		\\
\end{array}
\right.
\]

et les conditions aux bords :
\[
\left\{
\begin{array}{r c l c r}
\alpha U + U_x	&=&		0 	&	x = 0\\
W		- V_x 	&=&		0	&	x = 0\\
U_x	&=&		0 	&	x = 1\\
V_x 	&=&		0	&	x = 1\\
\end{array}
\right.
\]

\subsection{Adjoint}
Il nous faut maintenant obtenir l'adjoint.

\subsection{Algorithme}

A partir de ces étapes, on peut donner l'algorithme de résolution suivant :

\section{Analyse de la convergence}
Nous allons admettre la convergence du problème non bruité et montrer la	 convergence du problème bruité.

On appelle $x^{n, \delta}$ l'itération $n$ du problème bruité et on s'intéresse à $\|x^{n,\delta} - \hat{x} \|$. On peut écrire l'inégalité triangulaire :

\[
\|x^{n,\delta} - \hat{x} \| \leq \|x^{n,\delta} - x^n \| + \|x^n - \hat{x} \|
\]

On sait $\|x^n - \hat{x} \|$ tend vers zero puisqu'on a supposé la convergence pour le problème non bruité. Il nous reste à majorer le second terme.

On peut montrer que :
\[
\|x^{n,\delta} - \hat{x} \| \leq n\omega \delta + \|x^n - \hat{x} \|
\]

Et en supposant que $\hat{x} \in ImA^*$, on peut avoir :
\[
\|x^{n,\delta} - \hat{x} \| \leq n\omega \delta + \frac{C}{n \omega}
\]

Ce qui montre qu'il existe un nombre d'itération optimal qui dépend de la valuer de $\delta$. On a $n(\delta)=O(\sqrt{\delta})$.


\section{Résultats numériques}

Les résultats numériques sont particulièrement intéressant puisqu'ils ont des implications physiques intrinsèques. Nous allons étudier plusieurs configurations pour le problème à savoir :
\begin{itemize}
\item \textbf{La gamme de température} : nous étudierons le cas où la temperature est prise dans tout l'intervalle $[0;1]$, puis nous verrons le cas "réaliste" où la température est prise dans $[0.2;1]$.
\item \textbf{Le nombre d'itérations :} nous avons vu lors de l'étude de la convergence que pour des données bruitées, il existe un nombre d'itérations otpimal.
\item \textbf{L'erreur :} nous verrons que l'erreur est fonction du bruit.
\end{itemize}

Les données exactes sont obtenues en résolvant le problème directe avec un discrétisation très fine. Elles sont ensuite perturbées pour obtenir les données bruitées. Pour simplifier, on prend des valeurs constantes pour $a$.

\subsection{Gamme de température}

Les résultats obtenus sont meilleurs pour la gamme de température réaliste. Sur les graphique, on remarque que la solution calculée est moins bonne uniquement là en dehors de $[0.2;1]$.

\subsection{Convergence et erreur}

Dans la condition d'arr\^et de l'aglorithme, on a un paramètre $\tau$. On le fixe à une valeur faible $\tau = 2.1$.

On constate que le nombre d'itérations optimal dépend fortement de la valeur de $a$. Ceci montre que le problème est mal posé puisque seules les premières itérations sont proches de la solution.

Le choix de la valeur de $\omega$ est important et permet d'accélérer la convergence mais il faut cependant faire attention a ne pas prendre $\omega$ trop élevé si les résidus sont proches de $\tau \delta$.


\clearpage
\end{document}