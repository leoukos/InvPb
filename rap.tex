\documentclass[a4paper,10pt]{article}
\usepackage[utf8]{inputenc}
\usepackage[frenchb]{babel}
\usepackage[T1]{fontenc}

\usepackage{authblk}
\usepackage{frbib}
\usepackage{amsfonts}
\usepackage{amsmath}
\usepackage{amssymb}
\usepackage{array}
\usepackage{url}
\usepackage{hyperref}
\usepackage{textcomp}
\usepackage[all]{hypcap}
\usepackage[labelseparator=endash]{caption}
\usepackage[listofformat=parens]{subfig}
\usepackage{graphicx, color}
\usepackage{listings}
\lstset{language=C}
\lstset{morecomment=[l]{//}}
\lstset{basicstyle=\footnotesize}
\lstset{keywordstyle=\bf \color {blue},
commentstyle=\color{OliveGreen},
stringstyle=\color {red}}
\lstset{frame=single, rulesepcolor=\color{black}}
\usepackage{float}
\usepackage{algorithm}
\usepackage{algorithmic}
\usepackage{float}
\bibliographystyle{frplain}
\hypersetup{colorlinks,%
            citecolor=black,%
            filecolor=black,%
            linkcolor=black,%
            urlcolor=blue}

\newcommand{\LSTRESET}{
  \lstset{
    language=,
    basicstyle=\color{black},
    identifierstyle=\color{black}
}
}

\setlength{\affilsep}{0.2cm}
\author{François Deslandes}
\affil{Génie Mathématique 5ème année}
\affil{A l'attention de M. Caputo}
%\date{4 Décembre 2013\\}
\title{\Huge{Analyse non linéaire et problèmes inverses}\\
\LARGE{Problèmes inverses liés à la cristallisation de polymères.}\\
\vspace{10mm}
}

% EASY PARTIAL DERIVATIVE
\newcommand{\pd}[2]{\frac{\partial #1}{\partial #2}}

\DeclareMathOperator{\Rec}{Rec}
\DeclareMathOperator{\card}{card}

\begin{document}
\maketitle\thispagestyle{empty}
 
\newpage\null\thispagestyle{empty}\setcounter{page}{0}

\newpage
\tableofcontents
\newpage

\section{Introdution}
Nous nous intéressons dans cette étude à la cristallisation de polymères dans différentes conditions de formation. Ce problème présente un intér\^et particulier pour les spécialistes du domaine, d'une part parce que l'accès à des mesures directes est difficile, d'autre part parce que la qualité des résultats dépend de l'utilisation de paramètres physiques précis. Les mesures physiques directes disponibles sont bruitées, ce qui ajoute une difficulté supplémentaire qui impacte la qualité des résultats.


Nous utiliserons dans cette étude un modèle physique, en dimension 1, faisant intervenir deux équations aux dérivées partielles couplée, l'une elliptique et l'autre hyperbolique, modélisant l'évolution de la température et du degré de cristallisation. Ces deux équations dépendent de deux paramètres physiques dépendant fortement de la température,les taux d'accroissement et de nucléation, dont l'identification présente un intérêt particulier puisqu'ils ne peuvent pas être mesurés.


On souhaite produire un algorithme stable permettant d'identifier, en résolvant le problème inverse, le taux de nucléation.

\section{Modèle de base et problème d'identification}

Le problème modélisant la cristallisation de polymères est le suivant :
\[
\left\{
\begin{array}{r c l}
\frac{\partial T}{\partial t} &=& 
			D \frac{\partial ^2T}{dx ^2} 
			+ L \frac{\partial \xi}{\partial t} \\
					
\frac{\partial }{\partial t} (\frac{1}{\widetilde{G}(T) (1-\xi)}) &=& 
			\frac{\partial}{dx} (\frac{1}{\widetilde{G}(T) (1-\xi)})
			+ 2 \frac{\partial }{\partial t}(\widetilde{N}(T))
\end{array}
\right.
\]

Avec les conditions initiales suivantes :
\[
\left\{
\begin{array}{r c l}
T(x,0)		&=& 		T^ 0(x)		\\
\xi(x,0)		&=&		0			\\
\frac{\partial\xi}{\partial t}(x,0) 		&=&		0		\\
\end{array}
\right.
\]

et les conditions aux bords :
\[
\left\{
\begin{array}{r c l r}
\frac{\partial T}{\partial n}(x, t)		&=&		\alpha (T(x,t) -T^1(x,t)) & x \in \partial \Omega\\

\frac{\partial T}{\partial t}(x, t)	+ \widetilde{G}(T) \frac{\partial T}{\partial n}(x, t)	&=&		0	 & x \in \partial \Omega\\


\end{array}
\right.
\]

On cherche à identifier $\widetilde{N}(T)$,le taux de nucléation, connaissant $\widetilde{G}(T)$, le taux de croissance radial, $\xi(x, t_*)$, le degré de cristallisation, et $T$ la température sur les bords du domaine. Les constantes $L$ et $D$ sont des paramètres physiques correspondants respectivement à l'enthalpie de changement d'état et au coefficient de diffusion. $\alpha$ est le coeffient de convection thermique.

\section{Modèle simplifié  et problème d'identification}

Avant de pouvoir utiliser ce modèle, nous devons faire des simplifications, notamment pour avoir un modèle avec deux équations d'ordre 1 en temps. On va donc poser les changements de variables suivants :

%\pd{\xi}{t} & = & \widetilde{G}(T)(1-\xi)\omega  	\\
\[
\begin{array}{r c l}
w & = & \frac{1}{\widetilde{G}(T)(1-\xi)}\pd{\xi}{t} \\
\omega_t= & = &
	 \pd{}{x}(\frac{\widetilde{G}(T)}{1-\xi}\pd{\xi}{x}) + 2 \pd{\widetilde{N}(T)}{t}
\end{array}
\]


\[
\begin{array}{r c l}
u & = &  T \\
\end{array}
\]

\[
\begin{array}{r c l}
v & = & -ln(1-\xi) \\
v_t & = & \xi_t \frac{1}{1-\xi} \\
v_x & = & \xi_x \frac{1}{1-\xi} \\
\end{array}
\]

Par symétrie, on peut utiliser les conditions à droite suivante et on appliquera une transformation pour que $[a, \frac{a+b}{2}]$ soit envoyé sur $[0,1]$ :
\[
\begin{array}{r c l}
\pd{T}{x}(\frac{a+b}{2}, .) & = & 0 \\
\pd{\xi}{x}(\frac{a+b}{2}, .) & = & 0 \\
\end{array}
\]

Le problème initial devient :
\[
\left\{
\begin{array}{r c l}
u_t		&=&		D u_{xx} + L e^{-v} v_t		\\
v_t		&=&		a(u)w						\\
w_t		&=&		(a(u)v_x)_x + b(u)_t			\\
\end{array}
\right.
\]

Avec les conditions initiales suivantes :
\[
\left\{
\begin{array}{r c l}
u|_{t=0}		&=&		u^0		\\
v|_{t=0}		&=&		0		\\
w|_{t=0}		&=&		0		\\
\end{array}
\right.
\]

et les conditions aux bords :
\[
\left\{
\begin{array}{r c l c}
u_x(0, t)	&=&		-\alpha (u(0,t) - u^1(0, t)) 	\\
w(0, t)		- v_x(0, t)	&=&		0				 	\\
u_x(1,t)		&=&		0								\\
v_x(1,t)		&=&		0								\\
\end{array}
\right.
\]


Dans ce problème, on cherche $b(u)$, l'équivalent de $\widetilde{N}(T)$ et $a(u)$ est l'équivalent de  $\widetilde{G}(T)$.

Le problème peut \^etre exprimé, avec l'opérateur non linéaire $F$, sous la forme suivante :
\[
\begin{array}{r l c}
F(b)		&=&		(u^{\delta}_{B}, v^{\delta}_{*}) \\
\end{array}
\]

Où $u^{\delta}_{B}$ et $v^{\delta}_{*}$ sont des mesures bruitées.


\section{Méthode de résolution}
Le problème direct admet une solution forte unique pour des valeurs du paramètres $b$. Trouver les conditions sur $b$ pour assurer que le problème est bien posé est très technique et par conséquent, on suppose que le problème direct est bien posé. On peut définir l'opérateur $F$ formellement de la façon suivante :
\[
\begin{array}{c}
F:D(F) \subset V:={b|b\in0 H^1([u_1,u_2], b(u_2)=0)} \rightarrow L^2(I) \times L^2(\Omega) \\
b \rightarrow (u|_{x=0}, v|_{t=t_*})
\end{array}
\]
Où le domaine d'espace temps est défini par :
\[
\begin{array}{c}
Q := \Omega x I = (0,1) \times (0, t_*)
\end{array}
\]

Le problème inverse en revanche, et très certainement mal posé. Il est donc naturel d'envisager une méthode de régularisation. La méthode de régularisation de Tikhonov, très utilisée pour résoudre des problèmes inverses, cependant, elle n'est pas envisageable dans le cadre fixé puisque la fonctionnelle n'est pas nécessairement convexe. La fonctionnelle peut présenté de nombreux minima locaux ce qui rend la minimisation beaucoup plus complexe. Nous allons par conséquent avoir recours à une méthode itérative.

Pour trouver $b(u)$, on utilise l'itération de Landweber.
\[
\begin{array}{r l c}
b^{k+1} 	&=&	b^{k} +  \omega F'(b^k)*(F(b^k) - (u^{\delta}_{B}, v^{\delta}_{*}) \\
\end{array}
\]

Où $F'$ représente la dérivée de l'opérateur $F$ et $F*$ son adjoint.\\

Par la suite, un algorithme itératif basé en partie sur le calcul de l'adjoint de $F$ sera donnée pour trouver $b$ à partir d'un choix arbiraire de $b^0$.

\subsection{Linéarisé}

Dans un premier temps, l'opérateur $F$ est séparé en : $F=\Psi \circ  \Phi$. L'opérateur $\Psi$ se charge de conditions limites et est linéaire, ce qui permet de dire que : $F'=\Psi \circ  \Phi '$. On suppose que $\Phi '$ existe.
On va obtenir $F'(b)h=\Psi \circ  \Phi'(b) h$ en linéarisant le système. Pour cela, on suppose que $(u,v,w)$ est solution. On suppose ensuite que $(u+U, v+V, w+W)$ est proche de la solution. Avec $a(u+U)=a(u)+a'(u)U$ et $b(u+U)=b(u)+b'(u)U+h(u)$, $(U,V,W)=\Phi'(b) h$ satisfait le système suivant :

\[
\left\{
\begin{array}{r c l}
U_t		&=&		D U_{xx} + L e^{-v} (v_t V +V_t)		\\
V_t		&=&		a(u)W + a'(u)Uw						\\
W_t		&=&		(a(u)V_x)_x + (a'(u)Uv_x)_x + (b'(u)U)_t + h(u)_t v\\
\end{array}
\right.
\]

Avec les conditions initiales suivantes :
\[
\left\{
\begin{array}{r c l}
u|_{t=0}		&=&		0		\\
v|_{t=0}		&=&		0		\\
w|_{t=0}		&=&		0		\\
\end{array}
\right.
\]

et les conditions aux bords :
\[
\left\{
\begin{array}{r c l c r}
\alpha U + U_x	&=&		0 	&	x = 0\\
W		- V_x 	&=&		0	&	x = 0\\
U_x	&=&		0 	&	x = 1\\
V_x 	&=&		0	&	x = 1\\
\end{array}
\right.
\]

\subsection{Adjoint}
Il nous faut maintenant obtenir l'adjoint. Pour cela, on va séparer $F'(b)h$.
\[
\begin{array}{l c r}
F'(b)h &=& S \circ R \circ J h \\
\end{array}
\]

L'opérateur $J$ s'occupe d'envoyer $V$ sur $L^2([u_2, u_1])$. L'opérateur $R$ 
envoie $h(u) \in L^2([u_1, u_2])$ sur $h(u(x,t)) \in L^2(\Omega \times I)$.  L'opérateur $S$ s'occupe des conditions de bords et initiales.

Pour obtenir l'adjoint, on peut écrire :
\[
\begin{array}{l c r}
F'(b)^* &=& J^* \circ R^* \circ S^* \\
\end{array}
\]

En résolvant un système d'équations en $(\phi, \psi, \theta)$ issus du linéarisé, on peut obtenir le résidus :
\[
\begin{array}{l c r}
(q, r) &=& F(b) - (u_b^ \delta, v_*^ \delta) \\
\end{array}
\]

On déduit ensuite :
\[
\begin{array}{l c r}
S*(q, r) &=& a(u)\psi \\
\end{array}
\]

La méthode de calcul des adjoints de $R^*$ et $J^*$ est difficilement utilisable en pratique. On va donc utiliser une méthode de projection pour approximer $J^*R^*\phi$. Pour cela, on va utiliser la projection orthogonale $\widetilde{\phi_N})$ de $\widetilde{\phi}:=J^*R^*\phi$ dans un sous-espace $V_N$ de $H^1([u_1, u_2])$. Ainsi, on peut écrire $\widetilde{\phi_N})$ sous la forme :
\[
\begin{array}{l c r}
\widetilde{\phi_N} = \sum_{i=1}^{N} 
  \alpha_i p_i \\
\end{array}
\]

On est alors amené à résoudre un système linéaire que l'on peut rendre tri-diagnoal en prenant les $p_i$ comme des splines linéaires.


\subsection{Algorithme}

A partir de ces étapes, on peut donner l'algorithme de résolution suivant :
\begin{algorithm}
\caption{Indentification de $b$ par la méthode de Landweber}
\begin{algorithmic}
\STATE \text{Choisir une solution initiale $b^0$}
\WHILE{\text{$\| F(b^k) - (u^{\delta}_B, v^{\delta}_*)  \| > \tau \delta $}} 
	\STATE \text{Calculer $F(b^ k)$ en résolvant le problème initial.}
	\STATE \text{Evaluer $(u_b, v_*)$}.
	\STATE \text{Calculer le résidu $(q^{\delta}, r^{\delta})$}.
	\STATE \text{Calculer l'adjoint $S^*(q,r)=a(u)\psi$}.
	\STATE \text{Calculer $F^{'}(b^k)^*(q,r)=J^*R^*a(u)\psi$ par la méthode de projection}.
	\STATE \text{Calculer la nouvelle itération 
$\begin{array}{r l c}
b^{k+1} 	&=&	b^{k} +  \omega F'(b^k)*(F(b^k) - (u^{\delta}_{B}, v^{\delta}_{*}) \\
\end{array}$}
\ENDWHILE
\end{algorithmic}
\end{algorithm}
\newpage

La condition d'arr\^et est basée sur une borne dans $L^2$ du bruit sur les observations. La valeur de $\tau$ doit \^etre fixée supérieure à 2 d'après la théorie sur la convergence. Il est cependant difficile de connaître calculer la valeur selon cette théorie.


\section{Analyse de la convergence}
Nous allons admettre la convergence du problème non bruité et montrer la	 convergence du problème bruité.

On appelle $x^{n, \delta}$ l'itération $n$ du problème bruité et on s'intéresse à $\|x^{n,\delta} - \hat{x} \|$. On peut écrire l'inégalité triangulaire :

\[
\|x^{n,\delta} - \hat{x} \| \leq \|x^{n,\delta} - x^n \| + \|x^n - \hat{x} \|
\]

On sait $\|x^n - \hat{x} \|$ tend vers zero puisqu'on a supposé la convergence pour le problème non bruité. Il nous reste à majorer le second terme.

On peut montrer que :
\[
\|x^{n,\delta} - \hat{x} \| \leq n\omega \delta + \|x^n - \hat{x} \|
\]

Et en supposant que $\hat{x} \in ImA^*$, on peut avoir :
\[
\|x^{n,\delta} - \hat{x} \| \leq n\omega \delta + \frac{C}{n \omega}
\]

Ce qui montre qu'il existe un nombre d'itération optimal qui dépend de la valuer de $\delta$. On a $n(\delta)=O(\sqrt{\delta})$.


\section{Résultats numériques}

Les résultats numériques sont particulièrement intéressant puisqu'ils ont des implications physiques intrinsèques. Nous allons étudier plusieurs configurations pour le problème à savoir :
\begin{itemize}
\item \textbf{La gamme de température} : nous étudierons le cas où la temperature est prise dans tout l'intervalle $[0;1]$, puis nous verrons le cas "réaliste" où la température est prise dans $[0.2;1]$.
\item \textbf{Le nombre d'itérations :} nous avons vu lors de l'étude de la convergence que pour des données bruitées, il existe un nombre d'itérations otpimal.
\item \textbf{L'erreur :} nous verrons que l'erreur est fonction du bruit.
\end{itemize}

Les données exactes sont obtenues en résolvant le problème directe avec un discrétisation très fine. Elles sont ensuite perturbées pour obtenir les données bruitées. Pour simplifier, on prend des valeurs constantes pour $a$.

\subsection{Gamme de température}

Les résultats obtenus sont meilleurs pour la gamme de température réaliste. Sur les graphique, on remarque que la solution calculée est moins bonne uniquement là en dehors de $[0.2;1]$.

\subsection{Convergence et erreur}

Dans la condition d'arr\^et de l'aglorithme, on a un paramètre $\tau$. On le fixe à une valeur faible $\tau = 2.1$.

On constate que le nombre d'itérations optimal dépend fortement de la valeur de $a$. Ceci montre que le problème est mal posé puisque seules les premières itérations sont proches de la solution.

Le choix de la valeur de $\omega$ est important et permet d'accélérer la convergence mais il faut cependant faire attention a ne pas prendre $\omega$ trop élevé si les résidus sont proches de $\tau \delta$.


\clearpage
\end{document}