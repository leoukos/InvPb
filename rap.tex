\documentclass[a4paper,10pt]{article}
\usepackage[utf8]{inputenc}
\usepackage[frenchb]{babel}
\usepackage[T1]{fontenc}

\usepackage{authblk}
\usepackage{frbib}
\usepackage{amsfonts}
\usepackage{amsmath}
\usepackage{amssymb}
\usepackage{array}
\usepackage{url}
\usepackage{hyperref}
\usepackage{textcomp}
\usepackage[all]{hypcap}
\usepackage[labelseparator=endash]{caption}
\usepackage[listofformat=parens]{subfig}
\usepackage{graphicx, color}
\usepackage{listings}
\usepackage{float}
\bibliographystyle{frplain}
\hypersetup{colorlinks,%
            citecolor=black,%
            filecolor=black,%
            linkcolor=black,%
            urlcolor=blue}

\newcommand{\LSTRESET}{
  \lstset{
    language=,
    basicstyle=\color{black},
    identifierstyle=\color{black}
}
}

\setlength{\affilsep}{0.2cm}
\author{François Deslandes}
\affil{Génie Mathématique 5ème année}
\affil{A l'attention de M. Caputo}
%\date{4 Décembre 2013\\}
\title{\Huge{Analyse non linéaire et problèmes inverses}\\
\LARGE{Problèmes inverses liés à la cristallisation de polymères.}\\
\vspace{10mm}
}

\DeclareMathOperator{\Rec}{Rec}
\DeclareMathOperator{\card}{card}

\begin{document}
\maketitle\thispagestyle{empty}
 
\newpage\null\thispagestyle{empty}\setcounter{page}{0}

\newpage
\tableofcontents
\newpage

\section{Modèle de base et problème d'identification}

Le problème modélisant la cristallisation de polymères est le suivant :
\[
\left\{
\begin{array}{r c l}
\frac{\partial T}{\partial t} &=& 
			D \frac{\partial ^2T}{dx ^2} 
			+ L \frac{\partial \xi}{\partial t} \\
					
\frac{\partial }{\partial t} (\frac{1}{\widetilde{G}(T) (1-\xi)}) &=& 
			\frac{\partial}{dx} (\frac{1}{\widetilde{G}(T) (1-\xi)})
			+ 2 \frac{\partial }{\partial t}(\widetilde{N}(T))
\end{array}
\right.
\]

Avec les conditions initiales suivantes :
\[
\left\{
\begin{array}{r c l}
T(x,0)		&=& 		T^ 0(x)		\\
\xi(x,0)		&=&		0			\\
\frac{\partial\xi}{\partial t}(x,0) 		&=&		0		\\
\end{array}
\right.
\]

et les conditions aux bords :
\[
\left\{
\begin{array}{r c l r}
\frac{\partial T}{\partial n}(x, t)		&=&		\alpha (T(x,t) -T^1(x,t)) & x \in \partial \Omega\\

\frac{\partial T}{\partial t}(x, t)	+ \widetilde{G}(T) \frac{\partial T}{\partial n}(x, t)	&=&		0	 & x \in \partial \Omega\\


\end{array}
\right.
\]

On cherche à identifier $\widetilde{N}(T)$,le taux de nucléation, connaissant $\widetilde{G}(T)$, le taux de croissance radial, $\xi(x, t_*)$, le degré de cristallisation, et $T$ la température sur les bords du domaine.

\section{Modèle simplifié  et problème d'identification}

Par symétrie et avec des changements de variables, le problème devient :
\[
\left\{
\begin{array}{r c l}
u_t		&=&		D u_{xx} + L e^{-v} v_t		\\
v_t		&=&		a(u)w						\\
w_t		&=&		(a(u)v_x)_x + b(u)_t			\\
\end{array}
\right.
\]

Avec les conditions initiales suivantes :
\[
\left\{
\begin{array}{r c l}
u|_{t=0}		&=&		u^0		\\
v|_{t=0}		&=&		0		\\
w|_{t=0}		&=&		0		\\
\end{array}
\right.
\]

et les conditions aux bords :
\[
\left\{
\begin{array}{r c l c}
u_x(0, t)	&=&		-\alpha (u(0,t) - u^1(0, t)) 	\\
w(0, t)		- v_x(0, t)	&=&		0				 	\\
u_x(1,t)		&=&		0								\\
v_x(1,t)		&=&		0								\\
\end{array}
\right.
\]


Dans ce problème, on cherche $b(u)$, l'équivalent de $\widetilde{N}(T)$. Il peut \^etre exprimé, avec l'opérateur non linéaire $F$, sous la forme suivante :
\[
\begin{array}{r l c}
F(b)		&=&		(u^{\delta}_{B}, v^{\delta}_{*}) \\
\end{array}
\]

Où $u^{\delta}_{B}$ et $v^{\delta}_{*}$ sont des mesures bruitées.


\section{Méthode de résolution}

Pour trouver $b(u)$, on utilise l'itération de Landweber.
\[
\begin{array}{r l c}
b^{k+1} 	&=&	b^{k} +  \omega F'(b^k)*(F(b^k) - (u^{\delta}_{B}, v^{\delta}_{*}) \\
\end{array}
\]

Où $F'$ représente la dérivée de l'opérateur $F$ et $F*$ son adjoint.\\

Par la suite, un algorithme itératif basé en partie sur le calcul de l'adjoint de $F$ sera donnée pour trouver $b$ à partir d'un choix arbiraire de $b^0$.

\section{Ma question}
On veut calculer $F'(b)$.\\

Dans un premier temps, l'opérateur $F$ est séparé en : $F=\Psi \circ  \Phi$. L'opérateur $\Psi$ se charge de conditions limites et est linéaire, ce qui permet de dire que : $F'=\Psi \circ  \Phi '$. On suppose que $\Phi '$ existe.
A partir de ce moment, je ne comprend plus. Il est dit que par linéarisation directe on peut conclure que la dérivée $(U,V,W):=\Phi '(b)h$. satisfait le système suivant : \footnote{Ce sont les équations (2.4-7)}

\[
\left\{
\begin{array}{r c l}
U_t		&=&		D U_{xx} + L e^{-v} (v_t V +V_t)		\\
V_t		&=&		a(u)W + a'(u)Uw						\\
W_t		&=&		(a(u)V_x)_x + (a'(u)Uv_x)_x + (b'(u)U)_t + h(u)_t v\\
\end{array}
\right.
\]

Avec les conditions initiales suivantes :
\[
\left\{
\begin{array}{r c l}
u|_{t=0}		&=&		0		\\
v|_{t=0}		&=&		0		\\
w|_{t=0}		&=&		0		\\
\end{array}
\right.
\]

et les conditions aux bords :
\[
\left\{
\begin{array}{r c l c r}
\alpha U + U_x	&=&		0 	&	x = 0\\
W		- V_x 	&=&		0	&	x = 0\\
U_x	&=&		0 	&	x = 1\\
V_x 	&=&		0	&	x = 1\\
\end{array}
\right.
\]

\textbf{Je ne comprend pas comment il obtient les systèmes d'équations (il utilise clairement les équations de départ en dérivant mais en dérivant quoi et comment?). Je ne comprend pas non plus à quoi correspond le h(u).}\\

Il calcule ensuite l'adjoint $F'*$ en séparant l'opérateur $F'$. \textbf{Il obtient à nouveau un système d'équations issu du modèle simplifié. Je suppose que c'est la même technique qui est employée? } \footnote{Ce sont les équations (2.11-14)}


\clearpage
\end{document}